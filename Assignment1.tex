% This LaTeX was auto-generated from MATLAB code.
% To make changes, update the MATLAB code and export to LaTeX again.

\documentclass{article}

\usepackage[utf8]{inputenc}
\usepackage[T1]{fontenc}
\usepackage{lmodern}
\usepackage{graphicx}
\usepackage{color}
\usepackage{listings}
\usepackage{hyperref}
\usepackage{amsmath}
\usepackage{amsfonts}
\usepackage{epstopdf}
\usepackage{matlab}

\sloppy
\epstopdfsetup{outdir=./}
\graphicspath{ {./Assignment1_images/} }

\begin{document}

\begin{par}
\begin{flushleft}
Learning matrices and vectors: \href{https://ocw.mit.edu/courses/electrical-engineering-and-computer-science/6-094-introduction-to-matlab-january-iap-2010/assignments/MIT6_094IAP10_assn01.pdf}{https://ocw.mit.edu/courses/electrical-engineering-and-computer-science/6-094-introduction-to-matlab-january-iap-2010/assignments/MIT6\_094IAP10\_assn01.pdf}
\end{flushleft}
\end{par}

\vspace{1em}

\begin{par}
\begin{flushleft}
Scalar variables:
\end{flushleft}
\end{par}

\begin{matlabcode}
a = 10
\end{matlabcode}
\begin{matlaboutput}
a = 10
\end{matlaboutput}
\begin{matlabcode}
b = 2.5*10e23
\end{matlabcode}
\begin{matlaboutput}
b = 2.5000e+24
\end{matlaboutput}
\begin{matlabcode}
c = 2 + 3i
\end{matlabcode}
\begin{matlaboutput}
c = 
   2.0000 + 3.0000i

\end{matlaboutput}
\begin{matlabcode}
d = exp((2i*pi)/3)
\end{matlabcode}
\begin{matlaboutput}
d = 
  -0.5000 + 0.8660i

\end{matlaboutput}

\vspace{1em}

\begin{par}
\begin{flushleft}
Vector variables:
\end{flushleft}
\end{par}

\begin{matlabcode}
aVec = [3.14 15 9 26]
\end{matlabcode}
\begin{matlaboutput}
aVec = 
    3.1400   15.0000    9.0000   26.0000

\end{matlaboutput}
\begin{matlabcode}
bVec = [2.71 8 28 128]'
\end{matlabcode}
\begin{matlaboutput}
bVec = 
    2.7100
    8.0000
   28.0000
  128.0000

\end{matlaboutput}
\begin{matlabcode}
cVec = 5:-0.2:-5
\end{matlabcode}
\begin{matlaboutput}
cVec = 
    5.0000    4.8000    4.6000    4.4000    4.2000    4.0000    3.8000    3.6000    3.4000    3.2000    3.0000    2.8000    2.6000    2.4000    2.2000    2.0000    1.8000    1.6000    1.4000    1.2000    1.0000    0.8000    0.6000    0.4000    0.2000         0   -0.2000   -0.4000   -0.6000   -0.8000   -1.0000   -1.2000   -1.4000   -1.6000   -1.8000   -2.0000   -2.2000   -2.4000   -2.6000   -2.8000   -3.0000   -3.2000   -3.4000   -3.6000   -3.8000   -4.0000   -4.2000   -4.4000   -4.6000   -4.8000

\end{matlaboutput}
\begin{matlabcode}
dVec = logspace(0,1,100) % need help
\end{matlabcode}
\begin{matlaboutput}
dVec = 
    1.0000    1.0235    1.0476    1.0723    1.0975    1.1233    1.1498    1.1768    1.2045    1.2328    1.2619    1.2915    1.3219    1.3530    1.3849    1.4175    1.4508    1.4850    1.5199    1.5557    1.5923    1.6298    1.6681    1.7074    1.7475    1.7886    1.8307    1.8738    1.9179    1.9630    2.0092    2.0565    2.1049    2.1544    2.2051    2.2570    2.3101    2.3645    2.4201    2.4771    2.5354    2.5950    2.6561    2.7186    2.7826    2.8480    2.9151    2.9836    3.0539    3.1257

\end{matlaboutput}
\begin{matlabcode}
eVec = 'Hello'
\end{matlabcode}
\begin{matlaboutput}
eVec = 'Hello'
\end{matlaboutput}

\vspace{1em}

\begin{par}
\begin{flushleft}
Matric variables:
\end{flushleft}
\end{par}

\begin{matlabcode}
aMat = zeros(9) + 2
\end{matlabcode}
\begin{matlaboutput}
aMat = 
     2     2     2     2     2     2     2     2     2
     2     2     2     2     2     2     2     2     2
     2     2     2     2     2     2     2     2     2
     2     2     2     2     2     2     2     2     2
     2     2     2     2     2     2     2     2     2
     2     2     2     2     2     2     2     2     2
     2     2     2     2     2     2     2     2     2
     2     2     2     2     2     2     2     2     2
     2     2     2     2     2     2     2     2     2

\end{matlaboutput}
\begin{matlabcode}
aMat_ = 2 * ones(9)
\end{matlabcode}
\begin{matlaboutput}
aMat_ = 
     2     2     2     2     2     2     2     2     2
     2     2     2     2     2     2     2     2     2
     2     2     2     2     2     2     2     2     2
     2     2     2     2     2     2     2     2     2
     2     2     2     2     2     2     2     2     2
     2     2     2     2     2     2     2     2     2
     2     2     2     2     2     2     2     2     2
     2     2     2     2     2     2     2     2     2
     2     2     2     2     2     2     2     2     2

\end{matlaboutput}
\begin{matlabcode}
bMat = diag([1 2 3 4 5 4 3 2 1])
\end{matlabcode}
\begin{matlaboutput}
bMat = 
     1     0     0     0     0     0     0     0     0
     0     2     0     0     0     0     0     0     0
     0     0     3     0     0     0     0     0     0
     0     0     0     4     0     0     0     0     0
     0     0     0     0     5     0     0     0     0
     0     0     0     0     0     4     0     0     0
     0     0     0     0     0     0     3     0     0
     0     0     0     0     0     0     0     2     0
     0     0     0     0     0     0     0     0     1

\end{matlaboutput}
\begin{matlabcode}
cMat = reshape(1:100,[10,10])
\end{matlabcode}
\begin{matlaboutput}
cMat = 
     1    11    21    31    41    51    61    71    81    91
     2    12    22    32    42    52    62    72    82    92
     3    13    23    33    43    53    63    73    83    93
     4    14    24    34    44    54    64    74    84    94
     5    15    25    35    45    55    65    75    85    95
     6    16    26    36    46    56    66    76    86    96
     7    17    27    37    47    57    67    77    87    97
     8    18    28    38    48    58    68    78    88    98
     9    19    29    39    49    59    69    79    89    99
    10    20    30    40    50    60    70    80    90   100

\end{matlaboutput}
\begin{matlabcode}
dMat = nan(3,4)
\end{matlabcode}
\begin{matlaboutput}
dMat = 
   NaN   NaN   NaN   NaN
   NaN   NaN   NaN   NaN
   NaN   NaN   NaN   NaN

\end{matlaboutput}
\begin{matlabcode}
eMat = randi([-3,3],5,3)
\end{matlabcode}
\begin{matlaboutput}
eMat = 
     3     1    -2
     1     2     1
    -3     2    -3
     2    -1    -2
     3     1    -3

\end{matlaboutput}

\vspace{1em}

\begin{par}
\begin{flushleft}
Scalar equations:
\end{flushleft}
\end{par}

\end{document}
